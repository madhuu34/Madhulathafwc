\documentclass[a4paper,12pt]{article}
\usepackage{graphicx}
\usepackage{amsmath}
\usepackage[margin=1in]{geometry}
\usepackage{amsmath, amssymb}
\usepackage{float}
\usepackage{caption}
\usepackage{subcaption}
\usepackage{xcolor}
\usepackage{fancyhdr}
\usepackage{array}
\usepackage{float}
\usepackage{amsfonts}
\usepackage{graphicx} 
\geometry{a4paper, top=0.7in, left=1in, right=1in, bottom=1in}

\begin{document}
\thispagestyle{fancy}
\fancyhf{} 
\renewcommand{\headrulewidth}{0pt} 

\fancyhead[L]{
        \includegraphics[width=8cm, height=1.7cm]{comet.jpeg} 
        }
\fancyhead[R]{
    Name: G.MADHU LATHA \\
    Batch:COMET.fwc029 \\
    Date:
}
\vspace{1cm}
\begin{center}
    {\LARGE \textbf{\textcolor{black}{\\  GATE QUESTION \\ EEE 2010 Q54}}}
\end{center}
\vspace{-1cm}
\section*{\textcolor{cyan}{ \\Question :}}

\textbf{Statement for Linked Answer Questions 54 and 55.}

The L-C circuit shown in the figure has an inductance $L = 1 \text{ mH}$ and a capacitance $C = 10 \text{ \textmu F}$.

\begin{center}
\includegraphics[width=0.4\textwidth]{task51.png}
    \textit{(Circuit Diagram of an LC circuit with a switch, L, C, and an initial voltage across C)}
\end{center}
\textbf{Q.54} The initial current through the inductor is zero, while the initial capacitor voltage is $100 \text{ V}$. The switch is closed at $t = 0$. The current $i$ through the circuit is:

\begin{enumerate}
    \item[(A)] $5\cos(5\times10^3 t)\text{ A}$
    \item[(B)] $5\sin(10^4 t)\text{ A}$
    \item[(C)] $10\cos(5\times10^3 t)\text{ A}$
    \item[(D)] $10\sin(10^4 t)\text{ A}$
\end{enumerate}
Given the L-C circuit with an inductance $L = 1 \text{ mH} = 1 \times 10^{-3} \text{ H}$ and a capacitance $C = 10 \text{ \text{\text{\textnormal{\textmu}}F}} = 10 \times 10^{-6} \text{ F}$.
The initial capacitor voltage is $V_C(0) = 100 \text{ V}$ and the initial current through the inductor is $i_L(0) = 0$.
The switch is closed at $t = 0$.

For an L-C circuit, the angular frequency of oscillation is given by:
$$\omega = \frac{1}{\sqrt{LC}}$$
Substitute the given values of L and C:
$$\omega = \frac{1}{\sqrt{(1 \times 10^{-3} \text{ H})(10 \times 10^{-6} \text{ F})}} = \frac{1}{\sqrt{10 \times 10^{-9}}} = \frac{1}{\sqrt{100 \times 10^{-10}}} = \frac{1}{10 \times 10^{-5}} = \frac{1}{10^{-4}} = 10^4 \text{ rad/s}$$

Let $q(t)$ be the charge on the capacitor. The voltage across the capacitor is $V_C(t) = \frac{q(t)}{C}$.
The current $i(t)$ through the inductor is related to the capacitor voltage by $L \frac{di}{dt} = V_C(t)$.
Alternatively, we can define the charge on the top plate of the capacitor as $q(t)$. The initial charge is $q(0) = C V_C(0)$.
$$q(0) = (10 \times 10^{-6} \text{ F})(100 \text{ V}) = 1 \times 10^{-3} \text{ C}$$
Since the initial current through the inductor is zero, the charge on the capacitor is maximum at $t=0$. Therefore, we can model the charge as a cosine function:
$$q(t) = Q_0 \cos(\omega t)$$
where $Q_0 = q(0) = 1 \times 10^{-3} \text{ C}$.
So,
$$q(t) = (1 \times 10^{-3}) \cos(10^4 t) \text{ C}$$
The current $i(t)$ as defined in the diagram flows out of the top plate of the capacitor. Thus, the current is the negative derivative of the charge on the top plate:
$$i(t) = -\frac{dq}{dt}$$
$$i(t) = -\frac{d}{dt} \left( (1 \times 10^{-3}) \cos(10^4 t) \right)$$
$$i(t) = -(1 \times 10^{-3}) (-\sin(10^4 t)) (10^4)$$
$$i(t) = (1 \times 10^{-3} \times 10^4) \sin(10^4 t)$$
$$i(t) = 10 \sin(10^4 t) \text{ A}$$

To double-check using the inductor voltage:
The voltage across the inductor is $V_L(t) = L \frac{di}{dt}$.
The initial voltage across the capacitor is $V_C(0) = 100 \text{ V}$.
In an LC circuit, $V_C(t) = V_C(0) \cos(\omega t)$ if the current is zero at $t=0$.
So, $V_C(t) = 100 \cos(10^4 t)$.
By KVL, $V_L(t) + V_C(t) = 0$ (since there's no external source after closing the switch).
$V_L(t) = -V_C(t) = -100 \cos(10^4 t)$.
Now, $L \frac{di}{dt} = -100 \cos(10^4 t)$.
$\frac{di}{dt} = \frac{-100 \cos(10^4 t)}{L} = \frac{-100 \cos(10^4 t)}{1 \times 10^{-3}} = -10^5 \cos(10^4 t)$.
Integrate to find $i(t)$:
$i(t) = \int -10^5 \cos(10^4 t) dt = -10^5 \frac{\sin(10^4 t)}{10^4} + K$
$i(t) = -10 \sin(10^4 t) + K$.
Using the initial condition $i(0) = 0$:
$0 = -10 \sin(0) + K \implies K = 0$.
So, $i(t) = -10 \sin(10^4 t) \text{ A}$.

There is a sign discrepancy between the two methods due to the assumed direction of voltage drop and current flow conventions. However, if we consistently apply KVL and the current definition shown in the diagram, $i = -\frac{dq}{dt}$ when $q$ is the charge on the top plate, then $i(t) = 10 \sin(10^4 t) \text{ A}$. The provided options are positive amplitudes.

The solution that matches one of the options is $i(t) = 10 \sin(10^4 t) \text{ A}$.

The final answer is $\boxed{\text{(D)}}$.
\end{document}

